\documentclass[10pt,a4paper]{article}
\usepackage[utf8]{inputenc}
\usepackage[spanish]{babel}
\usepackage[hidelinks]{hyperref}
\usepackage{fancyhdr}
\usepackage{lettrine}
\usepackage{fancyvrb}
\usepackage{multicol}

\author{Víctor Manuel Fernández Castro}
\title{Simetrías en lengua \\ \Large Palíndromos y bifrontes}
\date{12 de octubre de 2016}

\setlength{\parskip}{6pt}
\fancyhead[L]{Víctor M. Fdez-Castro}
\renewcommand{\footrulewidth}{0.4pt}

\begin{document}
	\maketitle
	\thispagestyle{empty}
	
	\newpage
	\setcounter{page}{1}
	\tableofcontents
	
	% Capítulo 1 ---------------------------------------------------------------
	
	\newpage
	\pagestyle{fancy}
	\section{Introducción}
	
	\lettrine{L}{a simetría} es un rasgo característico de formas, sistemas o entidades, relacionada con su \textbf{invarianza} bajo ciertas transformaciones \cite{simetria}. Dada su definición tan abierta puede comprender diversos significados y encontrarse ejemplos de simetrías en los más diversos campos de la cultura.
	
	Cuando hablamos de simetrías, la idea más común es tal vez la simetría en su \textbf{forma visual}, refiriéndonos a geometría (simetría especular): existe un plano que equidista de cualesquiera pareja de puntos homólogos del objeto. \cite{simetria_especular}
	
	Un aspecto interesante de la simetría especular es su aparición conceptualmente en las palabras y su uso en literatura para escribir frases que se leen de la misma forma en los dos sentidos.
	
	Algunos ingeniosos autores han conseguido construir complejas oraciones que forman palíndromos. Es también muy conocido el juego de hablar pronunciando las palabras al revés de como se escriben.
	
	En este trabajo vamos a analizar algunas de estas ocurrencias en la lengua española. Para ello, además de buscar en Internet, vamos a producir un \textbf{programa informático} dedicado a buscar palabras de este tipo en el diccionario.
	
	% Capítulo 2 ---------------------------------------------------------------
	
	\section{Palíndromos}
	\subsection{Definición}
	
	Un palíndromo ---del griego \textit{palin dromein}, que significa ``volver atrás''--- es una palabra o una expresión que \textbf{se lee igual adelante que atrás}. Si se trata de un número se le conoce como capicúa. \cite{palindromo}
	
	Hay varias formas de definir matemáticamente la simetría en palabras, entendidas como cadenas de símbolos. Tal vez, la más sencilla sea la siguiente:
	
	Sea \textit{x} una cadena, y \textit{n} su longitud, entonces:
	
	\begin{equation}
	Palindromo(x,n) \equiv \forall i / i \in [1..n] \Longrightarrow x_i = x_{n-i+1}
	\end{equation}
	
	En otras palabras: una palabra es un palíndromo si su primer símbolo es igual que el último, el segundo que el penúltimo, etcétera. Hay un caso trivial de simetría: \textbf{cualquier palabra de longitud 1} es necesariamente palíndromo.
	
	También podemos definir una palabra simétrica por recursividad, ya que, si \textit{x} es un palíndromo, entonces la palabra que resulta de eliminar las letras primera y última es también un palíndromo:
	
	\begin{equation}
	Palindromo(x,n) \equiv n = 1 \lor Palindromo(x_{2..n-1}, n - 2)
	\end{equation}
	
	\subsection{Programa}
	
	Vamos a escribir un sencillo programa en lenguaje Python para trabajar sobre un archivo que contiene todo el vocabulario español. \cite{teoruiz}
	
	Para obtener resultados más interesantes, vamos a ignorar los palíndromos triviales (palabras de una sola letra) y también los signos de acentuación.
	
	Nuestra función recibe el nombre del archivo que contiene cada palabra en una línea, y analiza cada una de ellas, eliminando en primer lugar el último signo ---que corresponde a la marca de fin de línea--- y \textbf{sustituyendo las letras acentuadas} por sus equivalentes no acentuadas.
	
	Cada palabra que cumpla la condición de palindromía se añade a una lista, que se devuelve al final de la función. El resultado es el siguiente:
	
	\begin{Verbatim}
table = str.maketrans('áéíóú', 'aeiou')

def findpalindromes(path):
    file = codecs.open(path, 'r', 'utf-8')
    palindromes = []

    for line in file:
        word = line[:-1].translate(table)

        if len(word) > 1 and word == word[::-1]:
            palindromes.append(word)

    return palindromes
	\end{Verbatim}
	
	\subsection{Resultados}
	
	\noindent
	La aplicación anterior encontró 72 palabras en el lemario introducido:
	
	\begin{multicols}{4}
	\begin{enumerate}
		\item aba
		\item ababa
		\item aca
		\item acá
		\item afufa
		\item aga
		\item aja
		\item ajá
		\item ajajá
		\item ala
		\item alá
		\item alalá
		\item allá
		\item álula
		\item ama
		\item ana
		\item aña
		\item ananá
		\item anilina
		\item anona
		\item apa
		\item apá
		\item apócopa
		\item ara
		\item arra
		\item asa
		\item DVD
		\item efe
		\item eje
		\item ele
		\item elle
		\item eme
		\item ene
		\item eñe
		\item ere
		\item erre
		\item ese
		\item gag
		\item kayak
		\item ll
		\item mam
		\item nen
		\item nin
		\item nomon
		\item non
		\item obo
		\item oídio
		\item ojo
		\item orero
		\item oro
		\item oso
		\item ososo
		\item oto
		\item ovo
		\item pop
		\item radar
		\item rajar
		\item rallar
		\item rapar
		\item rasar
		\item rayar
		\item razar
		\item recocer
		\item reconocer
		\item rever
		\item rodador
		\item rotor
		\item ses
		\item sus
		\item TNT
		\item ujú
		\item yatay
	\end{enumerate}
	\end{multicols}

	Aunque no aparecían en el archivo fuente, encontramos también las siguientes palabras palíndromas:
	
	\begin{multicols}{4}
	\begin{itemize}
		\item aibofobia
		\item arenera
		\item arepera
		\item aviva
		\item salas
		\item seres
		\item somos
		\item sometemos
	\end{itemize}
	\end{multicols}
	
	% Capítulo 3 ---------------------------------------------------------------
	
	\section{Bifrontes}
	
	\subsection{Definición}
	
	De una forma parecida a los palíndromos, los bifrontes son palabras o frases que tienen distinto significado si se leen a izquierda o a derecha. \cite{bifronte} 

	También podemos entender estos artificios como pares de \textbf{expresiones simétricas} o espejadas, en tanto que podríamos verlas como si hubiera un espejo entre ellas.
	
	Podemos construir bifrontes para cualquier palabra, pero solo en algunos casos tendrán sentido. 
	
	\subsection{Programa}
	
	Hemos diseñado un \textbf{algoritmo de búsqueda} para buscar todos los bifrontes del vocabulario.
	
    Recorreremos la lista de palabras y, para cada una, recorreremos nuevamente el diccionario (a partir de esa posición) para buscar semi-simetrías. La eficiencia de esta técnica es:
    
    \begin{equation}
    \frac{N^2}{2}
    \end{equation}
    
    \newpage
    Teniendo en cuenta que la lista tiene aproximadamente 85\,900 palabras, tendríamos que hacer:
    
    \begin{equation}
    \frac{85\,900^2}{2}=3\,689\,405\,000 \ operaciones
    \end{equation}
    
    Para optimizar el rendimiento, clasificaremos en memoria las palabras según su \textbf{letra inicial}, así para cada palabra, buscaremos solo en el subconjunto de palabras que empiecen por la letra final de la palabra en cuestión. Esto nos permite obtener una eficiencia mejor:
    
    \begin{equation}
    \frac{N^2}{2} \cdot \frac{1}{27} = \frac{85\,900^2}{54} \approx 136\,644\,630 \ operaciones
    \end{equation}
    
    A pesar de que siguen siendo muchas operaciones, hemos bajado el tiempo de proceso de 45 minutos a 96 segundos aproximadamente, un tiempo mucho más razonable. El programa resultante es el siguiente:
    
    \begin{Verbatim}
table = str.maketrans('áéíóú', 'aeiou')
    
def findreverse(path):
    file = codecs.open(path, 'r', 'utf-8')
    words = []
    wordst = []
    indices = {}
    reverses = []
    i = 0
   
    for line in file:
        word = line[:-1]
    
        if len(word) > 1 and not (word[0] == '-' or word[-1] == '-'):
            wordt = word.translate(table)
            words.append(word)
            wordst.append(wordt)
   
            if wordt[0] not in indices:
                indices[wordt[0]] = [i, i + 1]
            else:
                indices[wordt[0]][1] = i + 1
    
            i += 1
    
    for i in range(len(words)):
        for j in range(*indices[wordst[i][0]]):
            if j > i and wordst[i] == wordst[j][::-1]:
                reverses.append((words[i], words[j]))
    
    return reverses
    \end{Verbatim}
    
    \subsection{Resultados}
    
    El programa ha encontrado 17 bifrontes:
    
    \begin{multicols}{2}
    	\begin{enumerate}
    		\item abetuna - anúteba
    		\item aca - acá
    		\item acera - areca
    		\item adala - alada
    		\item adaza - azada
    		\item adra - arda
    		\item adula - aluda
    		\item aja - ajá
    		\item ala - alá
    		\item alúa - aula
    		\item amina - ánima
    		\item ansa - asna
    		\item apa - apá
    		\item odio - oído
    		\item onagro - órgano
    		\item orto - otro
    		\item raedor - rodear
    	\end{enumerate}
    \end{multicols}

	Mención aparte la palabra inventada \textit{supercalifragilisticoespialidoso}, que aparece en el cine, en la película de \textit{Mary Poppins} y en la que la protagonista pronuncia al revés, produciendo un bifronte: \textit{osodilaipseocitsilicarfilacrepus}.
	
	% Capítulo 4 ---------------------------------------------------------------
	
	\section{Frases palíndromas}
	
	Del mismo modo que existen palabras que forman un palíndromo, también es posible construir oraciones completas de los más diversos temas que presentan simetría.
	
	Algunas, quizás demasiado enrevesadas, respetan la sintaxis del lenguaje aunque su semántica no tiene mucho sentido. A continuación enumeramos algunas de las frases más interesantes que hemos encontrado.
	
	\begin{itemize}
		\item A la catalana banal, atácala.
		\item Adivina ya te opina, ya ni miles origina, ya ni cetro me domina, ya ni monarcas, a repaso ni mulato carreta, acaso nicotina, ya ni cita vecino, anima cocina, pedazo gallina, cedazo terso nos retoza de canilla goza, de pánico camina, ónice vaticina, ya ni tocino saca, a terracota luminosa pera, sacra nómina y ánimo de mortecina, ya ni giros elimina, ya ni poeta, ya ni vida. --- Ricardo Ochoa.
		\item ¿Acaso hubo búhos acá? --- Juan Filloy.
		\item Allí por la tropa portado, traído a ese paraje de maniobras, una tipa como capitán usar boina me dejara, pese a odiar toda tropa por tal ropilla. --- Luis Torrent
		\item Allí si María avisa y así va a ir a mi silla.
		\item Amo la pacífica paloma.
		\item Amor a Roma.
		\item Ana lava lana.
		\item Anita lava la tina.
		\item Añora la Roña.
		\item Atar a la rata.
		\item Átale, demoníaco Caín, o me delata. --- usado por Julio Cortázar.
		\item Ateo por Arabia iba raro poeta. --- Juan Filloy.
		\item Dábale arroz a la zorra el abad.
		\item Eva ya hay ave.
		\item Isaac no ronca así.
		\item La ruta natural.
		\item La ruta nos aportó otro paso natural.
		\item Las Nemocón no comen sal.
		\item Nada, yo soy Adán. --- Guillermo Cabrera Infante.
		\item No Mara, sometamos o matemos a Ramón.
		\item No di mi decoro, cedí mi don. --- Juan Filloy.
		\item No lata, no: la totalidad arada dilato talón a talón.
		\item Oirás orar a Rosario.
		\item Se van sus naves.
		\item ¿Son mulas o cívicos alumnos?	
	\end{itemize}

	\section{Conclusión}
	
	Como explicó el Profesor D. Miguel Ortega Titos, y además fuimos descubriendo a lo largo del curso, el concepto de simetría es tan amplio que no se puede dar una definición exacta. Además encontramos ejemplos de esta característica en muchos aspectos cotidianos.
	
	Aunque la simetría en lengua es más conceptual que geométrica, pues reflejar exactamente una palabra solo en pocos casos sería legible ---y estaría a merced de la tipografía empleada---, esperamos haber aportado un nuevo punto de vista en algo que, sin duda, aún podríamos hablar y escribir más extensamente.
	
	% Bibliografía -------------------------------------------------------------
	
	\clearpage
	\addcontentsline{toc}{section}{Bibliografía}
	\bibliographystyle{acm}
	\bibliography{simetrias}
	
\end{document}